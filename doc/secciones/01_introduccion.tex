\chapter{Introducción}

Desde hace ya más de 40 años, el videojuego ha llegado a ser uno de los fenómenos 
culturales con más influencia en nuestra sociedad. Ya no es solo algo que solamente 
los niños y adolescentes disfrutan, sino también adultos, ya habiendo crecido con 
ellos o jugando por primera vez a uno de ellos hace solamente unos meses.

Y es cierto que el entretenimiento que ofrecen es variado, pero también fomentan 
otras cualidades, como la interacción social mediante grupos con los que se juega o 
solamente se habla de ellos, promoviendo habilidades como la resolución de 
problemas, la toma de decisiones o la cooperación, y no menos importante, 
estimulan la creatividad del jugador.

Todo esto ha llevado a lo que hoy en día es una industria que no para de crecer, y 
con ello una gran variedad de títulos que los jugadores pueden acceder. Pero, 
aunque todo esto es cierto, el precio de los videojuegos no es algo que podamos 
ignorar.

Es por ello que todos recurrimos a buscar ofertas en las distintas tiendas, tanto 
de terceros (Amazon, GAME, etc.) cómo las tiendas de las propias plataformas. A lo 
que se está optando actualmente es usar distintos medios, como portales web 
dedicados a ello o canales de comunicación. Pero hay una gran cantidad de estos, 
algunos especializándose en una plataforma en concreto, y puede que sea complicado 
encontrar la oferta que se busca.

\section{Descripción del problema}

Como se ha mencionado anteriormente, es posible encontrar ofertas por diversos 
métodos. Podemos hablar de distintos portales web que nos permiten encontrar todas 
las ofertas de las distintas tiendas, como también canales de comunicación en 
aplicaciones como Telegram o Discord. Estos canales de Telegram y servidores de 
Discord están compuestos por comunidades de usuarios que realizan aportes de las 
distintas rebajas de precio, algo que es muy útil para las distintas personas que 
quieren ampliar su colección o que buscan ese nuevo título al que echarle horas.

Pero aquí se encuentra un problema: \textbf{recopilar y organizar la información 
que se encuentra en estos sitios puede llegar a ser tediosa}. Al haber tantos 
sitios con información bastante dispersa y duplicada, llega a ser frustrante 
navegar por decenas de mensajes en distintos canales para encontrar lo que nos 
interese, por no hablar de la cantidad de tiempo invertido. Además, se podría 
mejorar para estar al tanto de las últimas ofertas de manera rápida y eficiente.

Por otra parte, la información de estos canales y servidores es muy dinámica. No 
podemos llegar a saber cuanto tiempo va a estar una oferta disponible, ya sea 
porque caduque o porque se agote el producto. Es cierto que si estamos atentos 
podemos llegar a enterarnos de la oferta, sin embargo, no todo el mundo tiene tiempo 
para estar constantemente encima de su dispositivo.

\section{Usuarios identificados}

Con la definición que hemos dado del problema, vamos a identificar a los usuarios 
que principalmente tienen este problema. Usaremos la metodología de ``Personas'' para 
esta tarea.

La metodología de ``Personas'' es una técnica ampliamente utilizada en el diseño de 
productos y servicios para crear representaciones ficticias de usuarios reales. El 
objetivo principal de esta metodología es comprender y definir las necesidades, 
comportamientos y características de los usuarios de un producto o servicio.

Siguiendo esta metodología hemos identificado dos tipos de usuarios: personas con 
poco presupuesto y coleccionistas. Vamos a desarrollarlas a continuación:

\subsection{Persona con poco presupuesto}

\begin{itemize}
    \item \textbf{Nombre}: Carlos
    \item \textbf{Edad}: 20 años
    \item \textbf{Ocupación}: Estudiante
    \item \textbf{Descripción}:
    \begin{itemize}
        \item Carlos es un apasionado de los videojuegos, pero como estudiante 
        universitario, tiene un presupuesto limitado para gastar en 
        entretenimiento. A menudo busca ofertas y descuentos para expandir su 
        colección de videojuegos sin gastar demasiado.
        \item Es usuario de plataformas de mensajería como Telegram y Discord para 
        estar al tanto de ofertas y promociones, aunque encuentra desafiante buscar 
        ofertas específicas en canales y servidores abiertos y dispersos.
        \item Valora la posibilidad de acceder de manera conveniente a una 
        recopilación de ofertas de videojuegos para buscar de manera sencilla los 
        títulos que le interesan.
    \end{itemize}
\end{itemize}

\subsection{Coleccionista}

\begin{itemize}
    \item \textbf{Nombre}: Maria
    \item \textbf{Edad}: 30 años
    \item \textbf{Ocupación}: Profesional en una empresa
    \item \textbf{Descripción}:
    \begin{itemize}
        \item María es una coleccionista apasionada de videojuegos y tiene un 
        trabajo a tiempo completo. A pesar de tener un presupuesto más amplio en 
        comparación que otras personas interesadas en el mundillo, su tiempo es 
        limitado debido a sus responsabilidades laborales y personales.
        \item Le encanta buscar ofertas y descuentos para completar su colección, 
        sin embargo, no siempre tiene tiempo para revisar canales y servidores de 
        Telegram o Discord.
        \item María valora la comodidad y la eficiencia. Busca una solución que le 
        permita acceder rápidamente a ofertas de videojuegos relevantes para ella.
        \item Aunque está dispuesta a gastar más en videojuegos, también busca 
        obtener un buen valor por su dinero y evitar pagar de más.
    \end{itemize}
\end{itemize}

\section{Definición de la solución}

Es por esto que este proyecto se centrará en \textbf{conseguir la centralización de 
información de las distintas ofertas y rebajas de videojuegos que se encuentran en 
los canales de comunicación anteriormente mencionados}, principalmente en Telegram.

Con esto, se pretende conseguir que los usuarios puedan acceder de manera más 
rápida y sencilla, pasando de tener que comprobar distintos canales a tener toda la 
información en un mismo sitio. Además, con esto no es necesario estar como miembro 
en cualquiera de estos canales, ya que solamente sería necesario tener acceso a 
dicho sitio.

Hemos decidido de centrarnos en Telegram como fuente por tener un número de 
usuarios mayor al que tiene Discord en este ámbito, encontrando entre ellos a 
Ofertas Juegos, el canal más grande de Telegram en cuanto a ofertas de videojuegos, 
contando con canales específicos de cada plataforma.

En resumen, la solución que queremos desarrollar consistirá en un sitio que 
contendrá la información de los distintos canales de Telegram manera centralizada 
para así conseguir que los usuarios puedan encontrar las ofertas de manera más 
rápida y eficiente.
