\chapter{Estado del arte}

Como se ha mencionado anteriormente, hay bastantes maneras de buscar ofertas de 
videojuegos, ya sea por portales web como \url{chollometro.com}, o por canales de 
comunicación, teniendo como ejemplo específico a \url{ofertasjuegos.es} en lo que 
respecta a canales de Telegram.

Es cierto que podemos buscar ofertas en ambos métodos (sobre todo en el ejemplo 
puesto para portales web dedicados a ello), pero aun así no podemos saber cuándo 
hay una oferta de algo que nos interesa al momento si no estamos revisando todo el 
rato el portal web, o puede que se pierda en todos los mensajes que se envían en el 
canal de Telegram.

Es por esto que este proyecto pretende coleccionar las ofertas desde un mismo sitio 
para que sea fácil explorar por ellas, buscando específicamente lo que nos interesa 
y sin la necesidad de pasar por todas esas ofertas que son irrelevantes para 
nosotros.

Esto ayuda a que el usuario pueda centrarse solamente en un único sitio y no tener 
la necesidad de estar entrando en varios portales web o canales de Telegram para 
encontrar ofertas.

Para el desarrollo del proyecto, necesitamos obtener la información de los 
distintos sitios de ofertas. En este caso, vamos a centrarnos primero en obtener 
los datos desde Telegram, que cuenta con una API para poder obtener los mensajes de 
un canal o chat, además de su uso extendido para los famosos bots de la aplicación.

Por tanto, es fácil encontrar librerías que permiten utilizar la API de Telegram de  
una manera más sencilla. Dentro de la propia documentación de Telegram 
podemos encontrar una lista curada 
\footnote{Bot API Library Examples \url{https://core.telegram.org/bots/samples}} de 
los que tienen más librerías.

Teniendo esto en cuenta, vamos a categorizar los lenguajes para elegir la mejor 
opción para el proyecto.

Las categorías serían:

\begin{itemize}
    \item \textbf{Facilidad de aprendizaje/entendimiento}: Queremos un lenguaje que 
    si no lo conocemos, podamos adaptarnos rápidamente a él.
    \item \textbf{Comunidad}: Si contamos no solamente que el lenguaje sea de
    rápida adaptación, sino que además tenga una comunidad extensa, nos será más 
    fácil resolver nuestras dudas.
    \item \textbf{Eficiencia del lenguaje}: Además de los puntos anteriores, es 
    importante elegir un lenguaje que sea eficiente en cuanto a recursos, ya que 
    podemos encontrarnos con que consuma demasiada memoria o haga que nuestra 
    factura sea un poco más alta de lo que esperábamos. Esto ya se ha explorado y 
    se puede encontrar una lista curada \cite{eficiencia} de los lenguajes.
\end{itemize}

Teniendo en cuenta estas categorías y la lista que nos proporcionan en la 
documentación de Telegram, hemos decidido utilizar Go como lenguaje de programación 
para nuestro proyecto.