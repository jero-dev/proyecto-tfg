\chapter{Implementación}

Como llegamos a mencionar en la implementación de planificación, el proyecto se ha 
dividido en varios hitos o Milestones (se les puede echar un vistazo en la 
\href{https://github.com/jero-dev/proyecto-tfg/milestones}{página del repositorio 
en GitHub}) en el que vamos a concentrar las distintas tareas que debemos de 
realizar para completar cada uno de ellos.

También habiendo realizado el análisis del problema mediante diseño dirigido por 
dominios, hemos podido dividir en tres los servicios de dominio para finalmente 
obtener nuestro producto mínimo viable. Estos serían:

\begin{itemize}
    \item \textbf{Obtención de información:} Necesitamos obtener las ofertas desde 
    los canales y servidores que podemos encontrar en Telegram y Discord 
    respectivamente.
    \item \textbf{Gestión de productos:} Necesitamos almacenar los productos que 
    los usuarios buscarán también.
    \item \textbf{Gestión de ofertas:} Necesitamos almacenar las ofertas que 
    dependerán del producto y la tienda. También queremos saber si la oferta está
    todavía vigente o no.
\end{itemize}

Con esto en mente, empecemos con el desarrollo del proyecto.
