\chapter{Implementación}

Ya teniendo todo planificado y sabiendo que metodologías vamos a seguir, 
empezaremos con la implementación de la solución. Lo primero, por supuesto, es 
haber creado nuestro \href{https://github.com/jero-dev/proyecto-tfg}
{repositorio en GitHub} donde subir todos los progresos que hagamos no solamente en 
el propio desarrollo de la solución, sino también en la redacción de la memoria del 
trabajo.

En cuanto a los otros elementos que hemos mencionado como las 
\href{https://github.com/jero-dev/proyecto-tfg/issues}{issues}, los 
\href{https://github.com/jero-dev/proyecto-tfg/milestones}{milestones} y 
\href{https://github.com/users/jero-dev/projects/1}{nuestro tablero Kanban}, se 
pueden encontrar en los enlaces que se encuentran en este párrafo.

Con todo esto ya realizado, es momento de empezar a identificar las historias de 
usuario que podemos encontrar gracias a la definición de personas que llegamos a 
realizar en el capítulo de introducción.

\section{Historias de usuario}

Habiendo identificado los distintos usuarios que hemos encontrado, podemos 
desarrollar las historias de usuario relacionadas o casos de uso que se pueden 
desarrollar para este proyecto. Para ello, hemos definido las siguientes historias:

\subsection{[HU01] Buscar videojuegos por nombre}

Como persona menor de 20 años/coleccionista, quiero poder buscar un videojuego por 
su nombre para obtener las distintas tiendas que lo venden y su precio, además de 
su enlace de compra para cada una.

\textbf{Condiciones de satisfacción}:

\begin{itemize}
    \item La solución debe de poder buscar un videojuego por su nombre.
    \item La solución debe de poder obtener las distintas tiendas que venden el 
    videojuego.
    \item La solución debe de poder obtener el precio de cada tienda.
    \item La solución debe de poder obtener el enlace de compra del videojuego para 
    cada tienda.
\end{itemize}

\subsection{[HU02] Obtener avisos automáticos acerca de un videojuego}

Como coleccionista, tendría en cuenta la opción de obtener avisos automáticos de la 
disponibilidad de un videojuego en concreto en el que tengo interés.

\textbf{Condiciones de satisfacción}:

\begin{itemize}
    \item La solución debe de ofrecer avisos automáticos de la disponibilidad de un 
    videojuego.
    \item En el aviso debe de aparecer el precio de cada tienda.
    \item En el aviso debe de aparecer el enlace de compra del videojuego para 
    cada tienda.
\end{itemize}

En nuestro caso nos centraremos principalmente en la primera historia de usuario, 
completando la segunda en caso de que tengamos tiempo suficiente para ello.

\section{Diseño de la aplicación}

Teniendo ya las historias de usuario, vamos a empezar a diseñar la solución. Para 
ello, seguiremos el diseño dirigido por dominios (en inglés, Domain Driven Design o 
DDD) \cite{ddd} para identificar de manera correcta el dominio del problema y poder 
así concentrarnos totalmente en él.

\begin{itemize}
    \item \textbf{Dominio del problema:} Tenemos como dominio del problema la 
    gestión de ofertas y promociones de productos (aunque nos centremos en 
    videojuegos). En cuanto a los conceptos (entidades) que hemos identificado, 
    tenemos solamente uno: el de \textbf{videojuego}. Aparte, tenemos también un 
    ``value object'' que sería el de \textbf{oferta} y un agregado de ambos que es 
    el de \textbf{producto}.
    \begin{itemize}
        \item \textbf{Videojuego}: Las propiedades de esta entidad serían un 
        \textbf{identificador único}, el \textbf{nombre} del producto y la 
        \textbf{plataforma} en la que se juega.
        \item \textbf{Oferta}: Las propiedades de este ``value object'' serían el 
        \textbf{precio} y el \textbf{enlace} a la tienda donde se encuentra la 
        oferta.
        \item \textbf{Producto}: Las propiedades de este agregado serían un 
        \textbf{videojuego} y una \textbf{lista de ofertas} que se han encontrado 
        para este.
    \end{itemize}
    \item \textbf{Contexto delimitado:} Para centrarnos en el problema, tenemos que 
    delimitar el contexto del mismo. Aquí tenemos claro que el contexto es la 
    gestión de ofertas.
    \item \textbf{Servicios de dominio:} Los servicios de dominio encapsulan la 
    lógica de negocio que no pertenece a ninguna entidad o valor. Aquí podemos sacar 
    un claro servicio de dominio: el procesamiento de los mensajes recibidos.
\end{itemize}

Teniendo el análisis realizado, podemos plasmarlo directamente a la estructura de 
nuestra aplicación. Primero, crearemos un directorio llamado \tt{entity} donde 
guardaremos la entidad \tt{VideoGame} con las propiedades mencionadas. Después, 
creamos otro directorio llamado \tt{value\_object} donde guardaremos el ``value 
object'' de \tt{Offer} también con sus propiedades apropiadas.

Este tipo de datos son básicos, sin ningún tipo de método fuera de un constructor y 
dos métodos de acceso para las propiedades de \tt{Offer}. Por ello, no vamos a 
crear pruebas unitarias para ellos.

Siguiendo con los conceptos que hemos identificado, queda el agregado de 
\tt{Product}. Este va a estar contenido en otro directorio llamado \tt{aggregate} 
en el que encontraremos tanto el tipo \tt{Product} como también las pruebas 
unitarias para el mismo divididos en dos ficheros: \tt{product.go} y 
\tt{product\_test.go}.

Necesitaremos guardar estos datos de alguna manera, así que añadiremos otro 
directorio llamado \tt{domain} que contendrá otro llamado \tt{product}. Aquí 
crearemos una interfaz que será la responsable de crear el contrato de cómo se debe 
de comportar un repositorio (por ello llamaremos al fichero \tt{repository.go}) 
para que no tengamos que depender de una tecnología en concreto. 

Para tener nuestro producto mínimamente viable, crearemos primero un repositorio en 
memoria que, si finalmente tenemos tiempo, podremos cambiar por uno que se conecte 
a una base de datos de nuestra preferencia. Es por eso que crearemos otro 
directorio llamado \tt{memory} con dos ficheros: \tt{memory.go} y su respectivo 
fichero de pruebas unitarias \tt{memory\_test.go}.

Finalmente llegamos a los servicios (o en este caso solamente servicio) de dominio. 
Crearemos un directorio llamado \tt{service} que contendrá un fichero llamado 
\tt{message_processor.go}. Al ser una lógica más compleja, dejaremos la 
implementación del servicio para la siguiente sección. También añadiremos el 
fichero principal de esta aplicación que llamaremos \tt{api.go} ya que es lo que 
al final será esta aplicación: una API.
