\chapter{Análisis del problema}
 
Habiendo descrito ya el problema, vamos a analizarlo. Para ello, seguiremos el 
diseño dirigido por dominios (en inglés, Domain Driven Design o DDD) \cite{ddd} 
para identificar de manera correcta el dominio del problema y poder así 
concentrarnos totalmente en él.

\begin{itemize}
    \item \textbf{Dominio del problema:} Tenemos como dominio del problema la 
    gestión de ofertas y promociones de productos (aunque nos centremos en 
    videojuegos). Los conceptos (entidades) que hemos identificado son:
    \begin{itemize}
        \item \textbf{Producto:} Bien o servicio que se ofrece en el mercado.
        \item \textbf{Precio:} Cantidad de dinero que se asigna a un producto.
        \item \textbf{Oferta:} Reducción del precio de un producto por tiempo 
        limitado.
        \item \textbf{Tienda:} Establecimiento donde se venden productos.
    \end{itemize}
    Habiendo comentado acerca de usuarios, los ejemplos de usuario que vamos a 
    seguir para el desarrollo del proyecto son:
    \begin{itemize}
        \item \textbf{Estudiante:} Usuario que no tiene un gran presupuesto para 
        comprar los productos que quiere y desea estar informado de las ofertas que 
        le permitan adquirirlos.
        \item \textbf{Fanático o coleccionista:} Usuario que tiene bastante 
        presupuesto en comparación al usuario anterior, pero que prefiere obtener 
        el mayor número de productos posibles por el mismo precio.
    \end{itemize}
    \item \textbf{Agregados:} Estos son conjuntos de entidades y/o valores que se 
    tratan como una sola unidad. En este caso, podemos tener el agregado de 'oferta 
    de producto', que se compondría de las entidades 'producto', 'oferta' y 'tienda'.
    \item \textbf{Contexto delimitado:} Para centrarnos en el problema, tenemos que 
    delimitar el contexto del mismo. Aquí tenemos claro que el contexto es la 
    gestión de ofertas.
    \item \textbf{Servicios de dominio:} Los servicios de dominio encapsulan la 
    lógica de negocio que no pertenece a ninguna entidad o valor. Aquí podemos sacar 
    tres claros servicios de dominio: 'obtención de información', 'gestión de 
    productos' y 'gestión de ofertas'.
    \item \textbf{Eventos de dominio:} Estos son sucesos significativos que ocurren 
    dentro del dominio. Aquí primordialmente tendríamos dos eventos: el de 'oferta 
    disponible' y el de 'oferta expirada'.
\end{itemize}
