\chapter{Análisis del problema}
 
Habiendo descrito ya el problema, vamos a analizarlo. Para ello, seguiremos el 
diseño dirigido por dominios (en inglés, Domain Driven Design o DDD) \cite{ddd} 
para identificar de manera correcta el dominio del problema y poder así 
concentrarnos totalmente en él.

\begin{itemize}
    \item \textbf{Dominio del problema:} Tenemos como dominio del problema la 
    gestión de ofertas y promociones de productos (aunque nos centremos en 
    videojuegos). En cuanto a los conceptos (entidades) que hemos identificado, 
    solamente tenemos uno: el de \textbf{oferta}. Estos se identificarán con un 
    \textbf{identificador único}, el \textbf{nombre} del producto, la 
    \textbf{plataforma} de este último, el \textbf{precio} de la oferta, el 
    \textbf{enlace} a la tienda con dicha oferta y la \textbf{fecha} se encontró 
    esa oferta.
    \item \textbf{Contexto delimitado:} Para centrarnos en el problema, tenemos que 
    delimitar el contexto del mismo. Aquí tenemos claro que el contexto es la 
    gestión de ofertas.
    \item \textbf{Servicios de dominio:} Los servicios de dominio encapsulan la 
    lógica de negocio que no pertenece a ninguna entidad o valor. Aquí podemos sacar 
    un claro servicio de dominio: el procesamiento de los mensajes recibidos.
\end{itemize}

Teniendo el análisis realizado, podemos plasmarlo directamente a la estructura de 
nuestra aplicación.
