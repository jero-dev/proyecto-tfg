\chapter{Conclusiones y trabajos futuros}

Aún teniendo ya una solución funcional y que cubre las necesidades de los usuarios, 
hemos comentado también que existen otras opciones que finalmente no hemos 
implementado por falta de tiempo, ya sea para desarrollar, investigar o probar.

Es por eso que se puede extraer de varias secciones de implementación algunas 
mejoras que se podrían realizar en el futuro, además de nuevas funcionalidades en 
la solución, ya que también tenemos otra historia de usuario identificada que no 
hemos llegado a implementar por no ser la más prioritaria.

\section{Conclusiones}

En conclusión, hemos conseguido desarrollar una solución que cubre las 
necesidades de los usuarios, que es la obtención de ofertas de manera rápida. Por 
supuesto, hemos identificado mejoras, pero al haber llevado a cabo un diseño de la 
aplicación con DDD, algunas de ellas se podrían efectuar de manera más sencilla que 
si no lo hubiéramos hecho.

El principal reto de este trabajo ha sido sobre todo el procesamiento de los 
mensajes, ya que estos, aunque siguen un patrón finalmente, están escritos en 
lenguaje natural, por lo que posiblemente en cuanto añadamos más canales de 
información, las expresiones regulares que se utilicen varíen.

Aunque no fueran el objetivo principal de este trabajo, también hemos aprendido 
como los bots o apps de servicios de mensajería pueden ser una herramienta muy 
potente no solo para la comunicación, sino también para la automatización o 
integración de servicios.

\section{Trabajos futuros}

Aquí podemos dividir las posibles mejoras en dos grupos: las que afectan a la 
integración de la solución y las que afectan a las funcionalidades de la misma.

\subsection{Integración de la solución}
\begin{itemize}
    \item \textbf{Uso de una base de datos}: Como llegamos a comentar en la sección 
    del diseño de la aplicación en el capítulo 4, ahora mismo estamos guardando y 
    accediendo a los datos almacenados en memoria. Esto para el inicio no supone un 
    problema para la funcionalidad, pero si se quiere que la aplicación sea 
    persistente, la base de datos es nuestra solución. Al haber implementado la 
    aplicación usando inyección de dependencias, esto nos permite cambiar la 
    implementación de manera sencilla sin ser un gran cambio en el código.
    \item \textbf{Uso de contenedores propios}: Al momento en el que hacemos el 
    despliegue de tanto los bots como la API, estamos utilizando contenedores que 
    vienen incluidos con el servicio de este. Esto nos permite desplegar de manera 
    rápida, aunque no tenemos un control total sobre el contenedor. Por eso, una 
    mejora sería crear nuestros propios contenedores con las especificaciones que 
    necesitemos.
    \item \textbf{Portal web o aplicación móvil}: También hemos comentado que por 
    obtener una solución funcional lo antes posible, hemos sacado un bot con el que 
    el usuario final puede interactuar. Aun así, no es la mejor solución en cuanto 
    a experiencia de usuario, ya que estamos más acostumbrados a utilizar o 
    portales web o aplicaciones móviles. Por eso, una mejora sería primero elegir 
    una de las dos opciones y desarrollarla.
\end{itemize}

\subsection{Funcionalidades de la solución}
\begin{itemize}
    \item \textbf{Más canales de comunicación}: Aunque, como mencionamos al 
    principio de este documento, nos hemos centrado sobre todo en los canales de 
    Telegram, aún queda por integrar la funcionalidad de Discord. No solamente eso, 
    sino que también se podría explorar la posibilidad de integrar la misma con 
    los nuevos canales de WhatsApp.
    \item \textbf{Avisos de disponibilidad}: Teniendo ya la funcionalidad 
    principal, que es la obtención de ofertas por parte de los usuarios, se podría 
    obtener avisos para cuando un título que el usuario está buscando se encuentre 
    disponible, que justamente cumpliría con la otra historia de usuario que hemos 
    identificado. Esta funcionalidad también podría ser una mejora de pago para los 
    usuarios, ya que los principales interesados serían los usuarios de perfil 
    coleccionista que, como comentamos en la sección de identificación de estos, 
    cuentan con un mayor presupuesto.
\end{itemize}